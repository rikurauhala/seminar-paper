\begin{abstract}

Conventional password-based single-factor authentication methods frequently result in user accounts being compromised and valuable data being stolen. Passwords can be obtained by malicious actors due to vulnerable code, password reuse, social engineering tactics and malware.

\textit{Multi-factor authentication} introduces an essential extra layer of security by requiring users to present two or more pieces of evidence called \textit{factors} to verify their identity. With multi-factor authentication, unauthorised access can be effectively thwarted even if one of the factors is compromised.

Despite the numerous benefits that multi-factor authentication has to offer, not all software developers are enthusiastic about integrating it into their applications. Implementing multi-factor authentication comes at the cost of extra work and increased complexity.

This seminar report examines the possible challenges associated with implementing multi-factor authentication from a software developer's perspective. Practical and theoretical solutions are discussed to address these challenges.

In this paper, multi-factor authentication is mostly discussed in the context of web, mobile and desktop applications, but some other examples of implementation are also mentioned. Due to the limited scope of this paper, only some challenges can be inspected in detail. The paper's contribution is offering software developers an overview of the topic and a starting point for implementing multi-factor authentication.

\end{abstract}
